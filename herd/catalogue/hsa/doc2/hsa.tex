\documentclass[a4paper]{article}
\usepackage[width=15cm,height=25.7cm]{geometry}
\usepackage{xspace}
\usepackage{color}
\usepackage[leqno]{amsmath}
\usepackage{amssymb}
\usepackage{array}
\usepackage{tabularx}
\usepackage{booktabs}
\usepackage{url}
\usepackage{hyperref}
\usepackage{breakurl}
\usepackage{graphicx}
\usepackage{subfigure}
\def\url@leostyle{%
  \@ifundefined{selectfont}{\def\UrlFont{\small\sf}}{\def\UrlFont{\small\sf}}}
\makeatother
\urlstyle{leo}
\usepackage[normalem]{ulem}
\input{version.tex}
\newcommand{\file}[1]{\texttt{#1}}
\newcommand{\afile}[1]{\ahref{#1}{\file{#1}}}
\renewcommand{\P}[1]{\ensuremath{P_{\ensuremath{#1}}}}
\newcommand{\opt}[1]{\texttt{#1}}
\newcommand{\myquoted}[1]{\textquotedbl{}#1\textquotedbl{}}
\usepackage{xspace}

%same notation policy for each instance of figure, thm, etc
\newcommand{\myfig}{Fig.~}
\newcommand{\mythm}{Thm.~}
\newcommand{\mycor}{Cor.~}
\newcommand{\mysec}{Sec.~}
\newcommand{\myapp}{App.~}
\newcommand{\wrt}{\mathit{w.r.t.}}
\newcommand{\eg}{\mathit{e.g.}}
\newcommand{\ie}{\mathit{i.e.}}
\newcommand{\cf}{\mathit{c.f.}}

%awful tricks
\let\mo\mathord

%les maths comme au lycee
%\let\implies\Rightarrow
\let\cimplies\Leftarrow
\let\iff\Leftrightarrow
\let\eset\emptyset
\newcommand{\transc}[1]{\mathop{{#1}^{+}}}
%\newcommand{\opt}[1]{\mathop{\mo{#1}?}}
\newcommand{\fst}{\engordnumber{1}}
\newcommand{\snd}{\engordnumber{2}}
\newcommand{\thrd}{\engordnumber{3}}
%iff
  \newcommand{\ns}{necessary and sufficient}

%def
  \newcommand{\dfn}[2]{{#1}~\triangleq~{#2}}
%BEGIN LATEX
  \let\eqdef\triangleq
%END LATEX
%type
  \newcommand{\type}[2]{{#1} : {#2}}

%rln
  \newcommand{\rlnt}[1]{rln~ #1}

%sets
\newcommand{\setcal}[1]{\mathcal{#1} }
\newcommand{\setbf}[1]{\mathbb{#1} }
\newcommand{\setfrak}[1]{\mathfrak{#1} }

  %events
  \newcommand{\e}{e}
  \newcommand{\me}{m}
  \newcommand{\re}{r}
  \newcommand{\we}{w}
  \let\w\we
  \newcommand{\ba}{b}
  \let\be\ba
  \newcommand{\ce}{c}

  %writes, reads, barriers
  \newcommand{\ud}{\_}
  \let\setset\setbf
  \newcommand{\evts}{\setset{E}}
  \newcommand{\mevts}{\setset{M}}
  \newcommand{\writes}{\setset{W}}
  \newcommand{\wtl}[1]{\setset{W}_{#1}}
  \newcommand{\wtlv}[2]{\setset{W}_{#1,#2}}
  \newcommand{\pw}{\operatorname{pw}}
  \newcommand{\reads}{\setset{R}}
  \newcommand{\rfl}[1]{\setset{R}_#1}
  \newcommand{\rflv}[2]{\setset{R}_{#1,#2}}
  \newcommand{\barriers}{\setset{B}}
  \newcommand{\commits}{\setset{C}}  
  \newcommand{\registerevents}{\setset{RG}}  
  \newcommand{\RSt}{\reads^{*}}
  \newcommand{\WSt}{\writes^{*}}

  %Marre de me tromper
  \let\stores\writes\let\loads\reads
  \let\fences\barriers
  %write to, read from
  \newcommand{\wt}[2]{write~to~ #1~ #2}
  \newcommand{\rfr}[2]{read~from~ #1~ #2}

  %procs
  \newcommand{\pr}{\operatorname{proc}}

  %locs
  \newcommand{\lo}{\ell}
  \newcommand{\loc}{\operatorname{loc}}

  %vals 
  \newcommand{\val}{\operatorname{val}}

  %event structures
  \newcommand{\ES}{\setcal{E}}
  \newcommand{\E}{E }

  %execution witnesses
  \newcommand{\XS}{\setcal{X}}
  \newcommand{\X}{X }
  %\newcommand{\VX}[3]{#1.valid~ #2~#3 }
  \newcommand{\VX}{\operatorname{valid}}

  %init,final
  \newcommand{\init}{\mathit{init} }
  \newcommand{\final}{\mathit{final} }

  %int
  \newcommand{\intra}{\mathit{int}}

  %ext
  \newcommand{\ext}{\mathit{ext}}

  %abcalc
  \newcommand{\abc}{\mathit{ab}}

  %check
  \newcommand{\AAcheck}{\operatorname{check_{\A}}}

  \newcommand{\Acheck}[1]{\operatorname{check_{#1}}}

  %ghb
  \newcommand{\Aghb}{\operatorname{ghb(\A)}}

  %architectures
  \newcommand{\A}{A}
  \newcommand{\Ae}{A^{\epsilon}}
  \newcommand{\AS}{\setcal{A}}

  %wmms
  \newcommand{\WS}{\setcal{W}}

%relations
\newcommand{\stacklabel}[1]
%{\stackrel{\smash{\scriptscriptstyle\#1}}}
{\stackrel{\smash{\scriptstyle\text{#1}}}}
  %any
  \newcommand{\rln}[1]{\ensuremath{\stacklabel{#1}{\rightarrow}}}
  \newcommand{\tc}[1]{\ensuremath{\stacklabel{{#1} +}{\rightarrow}}}
  \newcommand{\mrln}[1]{\ensuremath{\stacklabel{?#1}{\rightarrow}}}
  %misc
  \newcommand{\id}{\rln{id}}
  \newcommand{\locr}{\rln{loc}}
  \newcommand{\extr}{\rln{ext}}
  \newcommand{\intr}{\rln{int}}
  \newcommand{\rmwr}{\rln{rmw}}
  \newcommand{\amo}{\rln{amo}}
  %iico
  \newcommand{\iico}{\rln{iico}}
  \newcommand{\iicodata}{\rln{iico-data}}
  %po
  \newcommand{\po}{\rln{po}}
  \newcommand{\poloc}{\rln{po-loc}}
  \newcommand{\addr}{\rln{addr}}
  \newcommand{\data}{\rln{data}}
  
  \newcommand{\podrw}{\rln{podrw}}
\newcommand{\pposc}{\rln{ppo-sc}}
  \newcommand{\ppopso}{\rln{ppo-pso}}
  %potso
  \newcommand{\potso}{\rln{po-tso}}
  %ppotso
  \newcommand{\ppotsoc}{\mathit{ppo-tso}}
  \newcommand{\ppotso}{\rln{ppo-tso}}
  %ppo power
  \newcommand{\ppopow}{\rln{ppo-ppc}}

  %po_iico
%  \newcommand{\poiico}{\rln{po-iico}}

  %dp
  \newcommand{\dep}{\rln{dp}}
  \newcommand{\dpdr}{\rln{dpdr}}

  %rf
  \newcommand{\rf}{\rln{rf}}
  \newcommand{\grf}{\rln{grf}}
  \newcommand{\grfi}[1]{\rln{grf\ensuremath{_{\scriptscriptstyle#1}}}}
  \newcommand{\mrf}[1]{\rln{?rf\ensuremath{_{\scriptscriptstyle#1}}}}
  \newcommand{\rfsc}{\rln{grf\ensuremath{_{\scriptscriptstyle{2 \setminus 1}}}}}
  \newcommand{\mrfsc}{?(\rln{rf\ensuremath{_{\scriptscriptstyle{2 \setminus 1}}}})}
  \newcommand{\rfs}{rf\ensuremath{_{\scriptscriptstyle{2 \setminus 1}}}}
  \newcommand{\rfcands}{\rln{prf}}
  \newcommand{\caus}{causality~ }
  \newcommand{\wfrf}{wf~rf}

  %rfe
  \newcommand{\rfe}{\rln{rfe}}
  \newcommand{\mrfe}{\rln{?rfe}}
  \newcommand{\mrfei}[1]{\rln{?rfe\ensuremath{_{\scriptscriptstyle#1}}}}

  %rfi
  \newcommand{\rfi}{\rln{rfi}}
  \newcommand{\mrfi}{\rln{?rfi}}
  \newcommand{\mrfii}[1]{\rln{?rfi\ensuremath{_{\scriptscriptstyle#1}}}}

  %ws
  \newcommand{\ws}{\rln{ws}}
  \newcommand{\co}{\rln{co}}
  \newcommand{\coi}{\rln{coi}}
  \newcommand{\coe}{\rln{coe}}
  \newcommand{\iws}{\rln{wsi}}
  \newcommand{\ews}{\rln{wse}}
  \newcommand{\wsi}[1]{\rln{ws\ensuremath{_{\scriptscriptstyle#1}}}}
  \newcommand{\wsl}[1]{\rln{ws\ensuremath{_{\scriptscriptstyle#1}}}}
  \newcommand{\wscands}{\rln{pws}}
  \newcommand{\wfws}{wf~ ws}

  %fr
  \newcommand{\fr}{\rln{fr}}
  \newcommand{\ifr}{\rln{fri}}
  \newcommand{\efr}{\rln{fre}}
  \newcommand{\fri}{\rln{fri}}
  \newcommand{\fre}{\rln{fre}}
  %ab
  \newcommand{\abpow}{\rln{ab-ppc}}
  \newcommand{\ab}{\rln{ab}}
  \newcommand{\abi}[1]{\rln{ab\ensuremath{_{\scriptscriptstyle#1}}}}
%  \newcommand{\ab}[1]{\rln{ab\ensuremath{_{\scriptscriptstyle#1}}}}
  %absc
  \newcommand{\fenced}{\rln{fenced}}
  \newcommand{\fencedbase}{\rln{fenced-base}}
  \newcommand{\fencedcumul}{\rln{fenced-cumul}}
  \newcommand{\fsc}{\rln{fsc}}
  \newcommand{\absc}{\rln{absc}}

  %abtso
  \newcommand{\ftso}{\rln{ftso}}
  \newcommand{\abtso}{\rln{abtso}}

  %css
  \newcommand{\css}{\rln{css}}
  \newcommand{\cssi}[1]{\rln{css\ensuremath{_{\scriptscriptstyle#1}}}}

  %relax, safe
  \newcommand{\rrelax}{\rln{relax}}
  \newcommand{\safe}{\rln{safe}}

  %hb
  \newcommand{\hb}{\rln{hb-seq}}
  \newcommand{\hbi}[1]{\rln{hb\ensuremath{_{\scriptscriptstyle#1}}}}
  \newcommand{\hbpi}[1]{\rln{hb'\ensuremath{_{\scriptscriptstyle#1}}}}
  \newcommand{\hbd}{\rln{hb'\ensuremath{_{\scriptscriptstyle{d}}}}}
  \newcommand{\poi}[1]{\rln{hb''\ensuremath{_{\scriptscriptstyle#1}}}}
%  \newcommand{\hbpppi}[1]{\rln{hb'''\ensuremath{_{\scriptscriptstyle#1}}}}
    \newcommand{\hbpppi}[1]{\mathrel{R_{#1}}}
  %hbtso
  \newcommand{\hbtso}{\rln{hb-tso}}

  %ppo
  \newcommand{\ppoc}{\mathit{ppo}}
  \newcommand{\ppoi}[1]{\rln{ppo\ensuremath{_{\scriptscriptstyle#1}}}}
  \newcommand{\ppo}{\rln{ppo}}
  \newcommand{\ppoe}{\rln{ppo-ext}}
  \newcommand{\ppop}{\rln{ppo-proc}}
  \newcommand{\ppos}{\rln{ppo\ensuremath{_{\scriptscriptstyle{2 \setminus 1}}}}}

  %%

  %pio
  \newcommand{\pio}{\rln{po-loc}}

  %ghb
  \newcommand{\ghb}{\rln{ghb}}
  \newcommand{\ghbi}[1]{\rln{ghb\ensuremath{_{\scriptscriptstyle#1}}}}

  %lock
  \newcommand{\lock}{\rln{lock}}
  \newcommand{\locki}[1]{\rln{lock\ensuremath{_{\scriptscriptstyle#1}}}}

  %extras
  \newcommand{\rfreg}{\rln{rf-reg}}
  \newcommand{\dd}{\rln{dd}}
  \newcommand{\ddmem}{\rln{dd-mem}}
  \newcommand{\ctrl}{\rln{ctrl}}
  \newcommand{\ctrlisync}{\rln{ctrlisync}}
  \newcommand{\ctrlisb}{\rln{ctrlisb}}
  \newcommand{\isync}{\rln{isync}}
  \newcommand{\isb}{\rln{isb}}
  \newcommand{\ictrl}{\rln{ictrl}}
  %wf
  \newcommand{\wf}[1]{\operatorname{wf-#1}}
%criterions
  \newcommand{\wfe}{\operatorname{wf}}
  \newcommand{\acyclic}{\operatorname{acyclic}}
  \newcommand{\uniproc}{\operatorname{uniproc}}
  \newcommand{\thin}{\operatorname{thin}}
  \newcommand{\valid}{\operatorname{valid}}
%Barrier definitions.
  \newcommand{\noncumul}{\operatorname{non-cumul}}
  \newcommand{\acumul}{\operatorname{A-cumul}}
  \newcommand{\abcumul}{\operatorname{AB-cumul}}
  \newcommand{\abwrcumul}{\operatorname{AB-WR-cumul}}
  \newcommand{\acumulBis}{\operatorname{ext-A-cumul}}
%orders
  %so
  \newcommand{\so}{so}
  %tso
  \newcommand{\tso}{\rln{tso}}
  %ex
  \newcommand{\ex}{\rln{ex}}
  %hexa
  \newcommand{\hx}{hx}
  %linear strict order
  \newcommand{\lso}{\operatorname{total-order}}
  %partial order
  \newcommand{\parto}[2]{\operatorname{partial-order #1 #2}}
  %no intervening write
  \newcommand{\noi}[3]{\operatorname{no-intervening-write #1 #2 #3}}

%models
  %WMM
  \newcommand{\W}{W}
  \newcommand{\Wmm}{Wmm}
  %any
  \newcommand{\AWmm}{AWmm}
  \newcommand{\HW}[1]{HW(#1)}
  
  \newcommand{\must}{\rln{must}} 
  \newcommand{\may}{\rln{may}} 
  
  %machines
  \newcommand{\M}{M }
 
    %fully barriered
    \newcommand{\fb}{\operatorname{fb}}
    \newcommand{\wfb}{\operatorname{wfb}}

  %Power
  \newcommand{\Power}{\mathit{Power}}

  %SC
%  \newcommand{\dotop}{\hspace*{-0.3ex}\mathord{\texttt{.}}\hspace*{-0.5ex}}
  \newcommand{\dotop}{\mathord{\texttt{.}}}
  \newcommand{\wit}{\operatorname{wit}}
  \newcommand{\Sc}{\mathit{Sc}}
  \newcommand{\SC}{\mathit{SC}}%\xspace}
  \newcommand{\SCX}{\operatorname{\SC\dotop{}wit}}
%  \newcommand{\VSCX}[2]{\operatorname{\SC.valid #1 #2}}
%  \newcommand{\SCcheck}[2]{\operatorname{sc-check #1 #2}}
  \newcommand{\ScArch}{\Sc\dotop{}Arch}
  \newcommand{\ScWmm}{\Sc\dotop{}\Wmm}
%  \newcommand{\ScVX}[2]{\VX{\Sc} #1 #2}
  \newcommand{\seq}{\operatorname{seq}}
    %calculations
    %rfm,ws
    \newcommand{\scrfc}{\operatorname{\SC\dotop{}rf}}
    \newcommand{\scwsc}{\operatorname{\SC\dotop{}ws}}
  %po restricted to some events
  \newcommand{\WW}{\mathord{\mathit{WW}}}
  \newcommand{\RM}{\mathord{\mathit{RM}}}
  \newcommand{\MW}{\mathord{\mathit{MW}}}
  \newcommand{\WM}{\mathord{\mathit{WM}}}
  \newcommand{\WR}{\mathord{\mathit{WR}}}
  \newcommand{\MM}{\mathord{\mathit{MM}}}
  %TSO
  \newcommand{\Tso}{\mathit{Tso}}
  \newcommand{\Tsoe}{\mathit{Tso}^{\epsilon}}
 \newcommand{\TSO}{\mathit{TSO}}
  \newcommand{\Pso}{\mathit{Pso}}
  \newcommand{\Psoe}{\mathit{Pso}^{\epsilon}}
 \newcommand{\PSO}{\mathit{PSO}}
  \newcommand{\Rmo}{Rmo}
 \newcommand{\RMO}{RMO}
  \newcommand{\TSOX}{\operatorname{\TSO\dotop{}wit}}
  \newcommand{\VTSOX}[2]{\operatorname{\TSO.valid~exec #1 #2}}
  \newcommand{\PVTSOX}{\operatorname{ptso}}
  \newcommand{\TSOcheck}[2]{\operatorname{tso~check #1 #2}}
  \newcommand{\TsoArch}{\Tso.Arch}
  \newcommand{\RmoArch}{\Rmo.Arch}
  \newcommand{\PsoArch}{\Pso.Arch}
  \newcommand{\TsoWmm}{Tso.\Wmm}
  \newcommand{\TsoVX}[2]{\VX{\Tso} #1 #2}
   %calculations
    %rfm,ws
    \newcommand{\tsorfc}{\operatorname{\TSO\dotop{}rf}}
    \newcommand{\tsowsc}{\operatorname{\TSO\dotop{}ws}}
    %lo,ss
    \newcommand{\tsolo}[1]{R*~ #1~ }
    \newcommand{\tsoss}[1]{WW~ #1~}
    \newcommand{\tsosl}{WR }

  %DRF
  \newcommand{\Drf}{\operatorname{Drf}}
  \newcommand{\drf}{\operatorname{drf}}
  \newcommand{\drfreq}{\operatorname{wf-lock}}
  \newcommand{\compete}{\operatorname{compete}}
  \newcommand{\Cs}{\operatorname{Cs}}
  \newcommand{\cs}{\operatorname{cs}}

%instructions
\newcommand{\fno}{\operatorname{fno}}
\newcommand{\atom}{\operatorname{atom}}
%\newcommand{\rmw}{\operatorname{rmw}}
\newcommand{\rrmw}{\operatorname{rrmw}}
\newcommand{\res}{\operatorname{res}}
\newcommand{\taken}{\operatorname{taken}}
\newcommand{\free}{\operatorname{free}}
\newcommand{\locked}{\operatorname{locked}}
\newcommand{\Lock}{\operatorname{Lock}}
\newcommand{\Unlock}{\operatorname{Unlock}}
\newcommand{\import}{\operatorname{import}}
\newcommand{\export}{\operatorname{export}}
\newcommand{\lwarx}{{\tt lwarx}}
\newcommand{\stwcx}{{\tt stwcx.}}
\newcommand{\RES}{{\tt RES}}
\newcommand{\RESA}{{\tt RES\_ADDR}}
\newcommand{\CRO}{{\tt CR0}}

%%%Litmus (Luc)
%%basic format
%\let\prog\textsf
\let\as\texttt
\newcommand{\ltest}[1]{\textsf{\textbf{#1}}}
\newcommand{\atest}[1]{\ahref{#1.litmus}{\ltest{#1}}}
%%Some sets
\renewcommand{\L}{\mathcal{L}}
\newcommand{\VA}{\mathcal{V}}%for valid
\newcommand{\OB}{\mathcal{O}}%for observed

%% Conventional processor
\newcommand{\proc}[1]{\ensuremath{P_{#1}}}
%% Format graphs in columns
\newlength{\fmtlength}
\newcommand{\fmtgraphcol}[3]
{\setlength{\fmtlength}{\linewidth}%
\addtolength{\fmtlength}{-#2}%
\framebox{\resizebox{\fmtlength}{#3}{\input{#1.pstex_t}}}}
%%Format graph explicit width
\newcommand{\fmtgraph}[3]
{\resizebox{#2}{#3}{\input{#1.pstex_t}}}
%%Extra relations
\newcommand{\mfence}{\rln{mfence}}
\newcommand{\sfence}{\rln{sfence}}
\newcommand{\lfence}{\rln{lfence}}
\newcommand{\sync}{\rln{sync}}
\newcommand{\lwsync}{\rln{lwsync}}
\newcommand{\eieio}{\rln{eieio}}
\newcommand{\dmb}{\rln{dmb}}
\newcommand{\dsb}{\rln{dsb}}
\newcommand{\dmbst}{\rln{dmb.st}}
\newcommand{\dsbst}{\rln{dsb.st}}
\newcommand{\isfence}[1]{\textrm{is-#1}}
\newcommand{\fcd}[1]{\rln{fenced(#1)}}
\newcommand{\abk}[1]{\rln{ab-#1}}
%%Machine
\newcommand{\doko}{\texttt{doko}\xspace}
\newcommand{\hpcx}{\texttt{hpcx}\xspace}
%%%labels in figs
\newcommand{\lb}[4]{\((#1)\)\,#2#3#4}
\newcommand{\lbl}[3]{\lb{#1}{#2}{$\lo$}{#3}}
\newcommand{\lbi}[1]{\((#1)\) Import}
\newcommand{\lbe}[1]{\((#1)\) Export}
\newcommand{\lba}[1]{\lb{#1}{R/W}{x}{$v$}}

%%%For ``input cycles''
\newcommand{\longrln}[1]{\stacklabel{#1}{\longrightarrow}}
\newcommand{\Rfe}{\longrln{Rfe}}
\newcommand{\Fre}{\longrln{Fre}}
\newcommand{\Wse}{\longrln{Wse}}
\newcommand{\DpdR}{\longrln{DpdR}}
\newcommand{\DpdW}{\longrln{DpdW}}
\newcommand{\PodRW}{\longrln{PodRW}}
\newcommand{\PodWR}{\longrln{PodWR}}
%\newcommand{\cevt}[4]{(#1)#2\hspace*{0em}[#3]\hspace*{0em}\mbox{=}\hspace*{0em}#4}
\newcommand{\cevt}[4]{(#1)#2\hspace*{0em}#3\hspace*{0em}#4}
%%%
\let\textrel\textsf
\let\tid\texttt
\newenvironment{idtable}
{\begin{showcenter}\renewcommand{\rln}[1]{\tid{##1}}
\begin{tabular}{l@{\quad\quad}p{.2\linewidth}@{\quad\quad}p{.5\linewidth}}
\multicolumn{1}{c}{identifier} &
\multicolumn{1}{c}{name}  &
\multicolumn{1}{c}{description}
\\\hline}
{\end{tabular}\end{showcenter}}

\newenvironment{desctable}[2]
{\begin{showcenter}\renewcommand{\rln}[1]{\tid{##1}}
\begin{tabular}{l@{\quad\quad}p{.6\linewidth}}
\multicolumn{1}{c}{#1} &
\multicolumn{1}{c}{#2}
\\\hline}
{\end{tabular}\end{showcenter}}

%%%manual macros
\newcommand{\prog}[1]{\textsf{#1}}
\newcommand{\diy}{\prog{diy7}}
\newcommand{\diyone}{\prog{diyone7}}
\newcommand{\diycross}{\prog{diycross7}}
\newcommand{\litmus}{\prog{litmus7}}
\newcommand{\klitmus}{\prog{klitmus7}}
\newcommand{\herd}{\prog{herd7}}
\newcommand{\jingle}{\prog{jingle7}}
\newcommand{\dont}{\prog{dont7}}
\newcommand{\readRelax}{\prog{readRelax7}}
\newcommand{\mcycles}{\prog{mcycle7}}
\newcommand{\classify}{\prog{classify7}}
\newcommand{\norm}{\prog{norm7}}
\newcommand{\vbar}{\texttt{|}}
\newcommand{\cat}{\texttt{cat}}
\newcommand{\bell}{\texttt{bell}}
\newcommand{\safes}{\operatorname{Safe}}
\newcommand{\relaxs}{\operatorname{Relax}}
\newcommand{\allarch}{X86|PPC|ARM|AArch64|RISCV|LISA|C}
%%BNF
\newenvironment{remark}
{\par\begin{flushleft}\noindent\textbf{Remark}}
{\end{flushleft}}
\endinput
\newcommand{\T}[1]{\texttt{\color{Blue}#1}}
\newenvironment{syntax}
{\begin{center}%
\let\@T\T\let\@NT\NT
\renewcommand{\T}[1]{\addspace\@T{##1}\spacetrue}
\renewcommand{\NT}[1]{\addspace\@NT{##1}\spacetrue}
\begin{tabular}{r@{\quad}c@{\quad}l}}
{\end{tabular}\end{center}}
\newif\ifspace
\def\addspace{\ifspace \; \spacefalse \fi}
\def\brepet{\addspace\{}
\def\erepet{\}}
\def\boption{\addspace[}
\def\eoption{]}
\def\brepets{\addspace\{}
\def\erepets{\}^+}
\def\bparen{\addspace(}
\def\eparen{)}
\def\orelse{\mid \spacefalse}
\def\is{ & ::= & \spacefalse }
\def\alt{ \\ & \mid & \spacefalse }
\def\cutline{ \\ & & \spacefalse }
\def\sep{ \\[2mm] \spacefalse }


\newcommand{\cycle}[2][.25\linewidth]{\resizebox{#1}{!}{\input{#2.pstex_t}}}
\newcommand{\image}[2][]{\rotatebox{-90}{\includegraphics[#1]{#2.eps}}}
\newcommand{\img}[1]{\input{#1.pstex_t}}
\newcommand{\handletest}[1]{\ltest{#1}}
\newcommand{\xhandletest}[1]{\ltest{#1}}
\newenvironment{showcenter}
{\begin{center}}
{\end{center}}
%taken from the tutorial
% fix up fig2dev fonts to make actions in \sf
\makeatletter
% for the fig2dev version on P laptop
\gdef\SetFigFontNFSS#1#2#3#4#5{%
  \reset@font\fontsize{#1}{#2pt}%
%  \fontfamily{#3}\fontseries{#4}\fontshape{#5}%
  \fontfamily{\sfdefault}\fontseries{#4}\fontshape{#5}%
  \selectfont}%
% for the fig2dev version on P desktop
\gdef\SetFigFont#1#2#3#4#5{%
  \reset@font\fontsize{#1}{#2pt}%
%  \fontfamily{#3}\fontseries{#4}\fontshape{#5}%
  \fontfamily{\sfdefault}\fontseries{#4}\fontshape{#5}%
  \selectfont}
\makeatother


\title{A Cat tutorial: HSA Models}
\date{}
\author{Jade Alglave and Luc Maranget}
\begin{document}
\maketitle
\part{A reading of the HSA document}

\section{\label{coherence}Coherence}

For a given location~$L$,
the coherence order $\cohl{L}$ is defined as a total order on all loads and
stores to location~$L$. The ``single Coherent Order'' $\coh$ is
the union of all these orders for all locations.
In appendix~\ref{coh} we describe how to generate the set of all possible
$\coh$ ``orders'' in the \cat{} language.
For now, let us assume a variable \texttt{allCoh}
whose value is the set of all possible \coh{} orders.

The instruction \texttt{with $v$ from $S$} will, for each element~$e$ in $S$,
execute the rest of the model in an extended environment that binds
$v$ to~$e$. In our case of \coh{}, we write:
\begin{verbatim}
with coh from allCoh
\end{verbatim}
In effect, the construct performs a further enumeration of candidate
executions: the initial candidate is extended with one \coh{} order
bound to the variable~\texttt{coh}.

We can then  check the consistency of $\coh$ and~$\po$:
\begin{verbatim}
call consistent(coh,po) as CohPoCons
\end{verbatim}
See appendix~\ref{procedure} for the definition of the consistency check as
a procedure.

The other coherence check is the ``value of a load'' check, phrased
as: ``\emph{a load [\ldots] will always observe the most recent store in the coherent order of location~$L$}''.
Given the existence of initial writes, which are minimal amongst writes to
the same location in \coh{}, the above condition defines a relation from
writes to reads to the same location. This relation, \texttt{mincohWR},
is implemented in the \cat{} language as follows:
\begin{verbatim}
let cohWR = coh & (W * R)
let cohWW = coh & (W * W)
let mincohWR = cohWR \ (cohWW; cohWR)
\end{verbatim}
Where ``\verb+&+'' is intersection, ``\verb+*+'' is cartesian product
and ``\verb+\+'' is (set or relation) difference.
Then it remains to check that $\rf$ equals \texttt{mincohWR}.
We proceed by checking double inclusion:
\begin{verbatim}
(* Relation a includes relation b, ie b(x,y) => a(x,y) *)
procedure includes(a,b) = empty b \ a end
procedure equals(a,b) =
  call includes(a,b)
  call includes(b,a)
end

call equals(rf,mincohWR) as LoadCons
\end{verbatim}
Notice another usage of the difference operator on relations.

Finally, we may view the atomicity of RMW constructs as a coherence property
\marginpar{This property does not appear formally in the document?}
This property states that there cannot be any write in-between the read
and the write performed by a RWM operation.

There are two possible semantics of RMW ``operations'' as ``events''.
\begin{enumerate}
\item
A C-like semantics would consider RMW events that are both read and write
events. We would then state the atomicity of RMW's as follows:
\begin{verbatim}
let RMW = R & W
let rmw = id & (RMW * RMW)
let cohRW = coh & (R * W)
empty rmw & (cohRW;cohWW) as RmwCons
\end{verbatim}
Where \texttt{id} is the pre-defined identity relation on events.
As a consequence the relation \texttt{rmwid} is the identity restricted
to RMW events.
However, this condition is useless. Indeed, by construction,
\texttt{coh} is a total order and is thus irreflexive.
As a consequence, \texttt{cohRW;cohWW} which is included in \coh{}
cannot intersect the identity.
\item Or, and we prefer this solution, a RMW operation is represented
by two events a read and a write, which are related by a pre-defined
\texttt{rmw} relation.
We would then state the atomicity of RMW's as follows:
\begin{verbatim}
empty rmw & (cohRW;cohWW) as RmwCons
\end{verbatim}
In that case, we cannot get rid of the \texttt{RmwCons} check,
as \texttt{rmw} now relate different events.
\end{enumerate}

\section{No values out-of-thin-air, dependencies}
The document defines the ``local dependence order'' $\ldo$ informally
as the union of data, address and control dependencies (Sec 3.8).
Our simulator \herd{} provides pre-defined relations for those three
``local dependencies'' relations.
Hence we write
\begin{verbatim}
let ldo = data | addr | ctrl
\end{verbatim}
We notice that control dependencies to reads are part of \ldo.

The document  then define the ``global dependence order'' as
the irreflexive transitive closure of \ldo{} union \coh{},
which we would write \verb!let gdo = (ldo|coh)+!.
The document then states
``\emph{By rule, there cannot be a cycle in in \gdo{}}'',
which we interpret as a requirement: \gdo{} must be acyclic.
The intent of the global dependence order clearly is to rule out
values out of thin air, as illustrated by
the typical example ``\ltst{lb+ldos}'':
\begin{center}\fmt{lb+ldos}\end{center}

However, including the complete~\coh{} in the definition of~\gdo{} looks too
strong: the requirement will then for instance forbid \ltst{wrc+ldos}.
\begin{center}\hspace*{-2cm}\fmt{wrc+ldos}\end{center}

We  think, given the intent to forbid values out of thin air,
that \gdo{} should include a fragment of \coh{}: the \rf{} relation
when source and target belong to different units:
\begin{verbatim}
let rfe = rf & ext
let gdo = ldo | rfe
acyclic gdo as GdoCons
\end{verbatim}
In the code above, we first define the \rfe{} relation (\rf{} from different
units) by intersecting \rf{} and the pre-defined \ext{} relation that relates
all events from different units.
Also notice that we do not apply transitive closure to \gdo{}, as this
does not impact the acyclicity of a relation. Furthermore,
the \gdo{} relation will no longer appear in the remainder of the model.


\section{\label{sso}Scoped synchronization order}
The documents defines a hierarchy of five scope levels:
system, agent, work-group, wave and work-item,\footnote{We use some abbreviations:``wg'' for work-group and ``wi'' for and work-item.}
here listed from top to bottom.
The \herd{} simulator handles scopes by the means
of various objects: tags, scope relations,
and set of annotated events.
Those are defined by reading a prelude ``bell'' file,
supplied to \herd{} with the \texttt{-bell} option.
Appendix~\ref{bell} details the bell file for HSA.

Each test features a ``scope tree'' definition that specifies how
the test  units are organized according to the scope hierarchy.
Consider for instance the test \ltst{isa2}:
{\small
\begin{verbatim}
Bell isa2
{ 2:r1=-1; }
 P0                     | P1                         | P2                          ;
 w[] x 53               | r[atomic,scacq,agent] r0 y | r[atomic,scacq,system] r0 z ;
 w[atomic,screl,wg] y 1 | bne r0, 1, Exit1           | bne r0, 1, Exit2            ;
                        | w[atomic,screl,system] z 1 | r[] r1 x                    ;
                        | Exit1:                     | Exit2:                      ;
scopes: (agent (wg 0 1) (wg 2))
exists (1:r0=1 /\ 2:r0=1 /\ 2:r1=0)
\end{verbatim}
}
This test has three units, \myth{0}, \myth{1} and~\myth{2}.
According to the scope tree \verb+(agent (wg 0 1) (wg 2))+,
the first two units \myth{0} and \myth{1} are in the same work-group,
while the last unit~\myth{2} is in a different work-group.
Then, all units are in the same agent.
Some scopes remain unspecified: they implicitly contain only one
item --- \emph{e.g.} the system scope contains one agent, while
there are three work-item scopes that contain one unit each.

Hence, at runtime, there will be two ``scope instances'' of the work-group
level. Those scope instances are available to \cat{} model as a ``scope
relations'' whose names are the names of the corresponding scope levels.
A scope relation is an equivalence relation that relates events whose units
are in the same scope instance of the given level.
For instance, figure~\ref{isa2scopes} pictures the \wg{} relation
for the \ltst{isa2} test.\footnote{Reflexivity edges are omitted for clarity.}
\begin{figure}
\caption{\label{isa2scopes} The \wg{} scope relation of the \ltst{isa2} test}
\begin{center}\hspace*{-2cm}\fmt{isa2+scopes}\end{center}
\end{figure}
We could also have pictured the \agent{} relation. We refrain from doing so,
as there is a single agent scope instance that contains all units.
Thus, the \agent{} relation is the total relation over events,
and picturing it would clobber the diagram.

It is to be noticed that the bell file must define
an enumeration of name \texttt{scopes},
whose tags are the name of scope levels:
\begin{verbatim}
enum scopes = 'system || 'agent || 'wg || 'wave || 'wi
\end{verbatim}
Then, given an expression~$e$ whose value is the
tag \texttt{'\textit{lvl}}, the primitive \texttt{tag2scope($e$)} returns
the scope relation \textit{lvl}.

Scopes also appear as event annotations. Such annotations are the ones
of the instructions that generate them.
Sets of annotated events are also defined from bell file contents,
and are available to \cat{} models as pre-defined variables.
The name of one of those variables is the name
of the annotation with uppercase initial letter.
For instance, on figure~\ref{isa2scopes}, we have
$\texttt{Wg} = \{ b \}$,
$\texttt{Agent} = \{ c \}$,
$\texttt{System} = \{ d,e \}$.
Like scope relations, sets of annotated events can be accessed
from the scope tag by a primitive, \verb+tag2set+.
For instance, in our \ltst{isa2} example,
the expression \verb+tag2set('system)+ will evaluate to the set
of events $\{d,e\}$.

Section 3.9 ```Scoped synchronization order'' of the HSA document
states that two operations may ''\emph{both specify (directly or indirectly
through scope inclusion) scope instance~$S$}''.
We first notice that ``indirectly'' applies to the event annotations,
not to the scope instance that is fixed.
For instance, a memory operation whose scope annotation is agent also
specifies work-group.
More generally, the scope annotation~\textit{lvl} takes effect on
all scopes at levels \emph{lower} than~\textit{lvl}.

In practice, we shall define equivalence relations
\texttt{same-\textit{lvl}}, where \textit{lvl} is a scope level.
Two operations are related by \texttt{same-\textit{lvl}}
when they specify a common instance at level~\textit{lvl} in the HSA document
sense.
This not only means that the two operations
are related by the scope relation \textit{lvl} (\emph{i.e.}
they do belong to one common scope instance at level~\textit{lvl}),
but also that they are annotated by scope~\textit{lvl} or higher,
so that they take effect at level~\textit{lvl}.

Hence, we first define a function \texttt{all-events} that takes
a (scope) tag as argument and returns all the events annotated by
this scope or higher:
\begin{verbatim}
let rec all-events(tag) = match tag with
|| 'system -> tag2events(tag)
|| _ -> tag2events(tag) | all-events(wider(tag))
end
\end{verbatim}
The (omitted) \texttt{wider} function is defined in the bell file:
it simply returns
the tag that is immediately above its tag~argument.
That is, it associates \texttt{'system} to~\texttt{'agent}, \texttt{'agent}
to~\texttt{'wg}, etc.

Following our discussion, we then
define the function~\texttt{same-instance} that takes a scope
tag~\texttt{'\textit{lvl}} as argument and returns the
relation~\texttt{same-\textit{lvl}}:
\begin{verbatim}
let same-instance(lvl) =
 let evts = all-events(lvl) in
 tag2scope(lvl) & (evts * evts)
\end{verbatim}
The function simply performs the intersection of the scope relation
and of the cartesian product on events annotated with
scope level~\textit{lvl} or higher. Figure~\ref{isa2same} pictures two of
those ``same'' relations.\footnote{Reflexivity edges are omitted for clarity.
In particular $e \same{wg} e$ is omitted.}
\begin{figure}
\caption{\label{isa2same}The \same{wg} and~\same{agent} relations of the \ltst{isa2} test.}
\begin{center}\hspace*{-2cm}\fmt{isa2+instances}\end{center}
\end{figure}

We are now ready to define scope synchronization orders, which basically
formalize release-acquire synchronization, with scope restrictions.
From \herd{} pre-defined sets of annotated events we build a few
relevant sets:
\begin{verbatim}
let Release = Screl | Scar
let Acquire = Scacq | Scar
let Synchronizing = Acquire | Release (* Could be Screl|Scacq|Scar *)
\end{verbatim}
Notice that the \texttt{screl}, \texttt{scacq} and~\texttt{scar}
annotations apply to atomic operations and to fences only. As a consequence,
the above sets regroup atomic operations and fences only. 
We then directly follow the description of Sec. 3.9, up to a few innocent
changes:
\begin{verbatim}
let acq-rel =
  ((W & Release) * (R & Acquire)) & coh
| ((F & Release) * Acquire) &
  ((po & (_ * W)); coh; (po? & (R * _)))
| (Release * (F & Acquire)) &
  ((po? & (_ * W)); coh; (po & (R * _)))

let sso s = same-instance(s) & acq-rel  
\end{verbatim}
\marginpar{The proposed model enforces atomic accesses for the intermediate
$A$ and~$B$ operations such that $A \coh B$ in the fence-to-fence case.
Although this is a race, the HSA document does not specify this atomicity
condition.}
The definition of \texttt{acq-rel} regroups the three top-level items
of the description of the HSA document.
It uses set and relation constructs intensively, most of which have
already been introduced, except the option postfix
operator~\texttt{$e$?} that yields,
\texttt{$e$|id} (\emph{i.e.} $e$ union the identity relation),
and the universe set ``\verb+_+'' that contains all events.
Our (minor) changes are:
\begin{itemize}
\item Some simplifications apply,
as our \texttt{Acquire} and \texttt{Release} sets
contain synchronizing operations~only.
\item Similarly, we make no specific provision for
RMW events as they either belong to the pre-defined sets of reads and writes,
\verb+R+ and \verb+W+, or are represented by a read event (in~\verb+R+)
and a write event (in \verb+W+) --- See our Sec.~\ref{coherence}.
\item We have made the definition a bit more symmetric (and redundant)
by having the third item to produce fence-to-fence order,
as the second item does.
\item We have factored out the condition ``\emph{$X$ and~$Y$ both specify the same instance~$S$}'' (implemented by the call ``\verb+same-instance(s)+'').
\end{itemize}

Finally the HSA-happens-before order~\hhb{}
is defined by following the HSA~document,
as the transitive closure of the union of the program order and of
the union of scope synchronization order for all scopes.
\begin{verbatim}
let union-scopes f = fold (fun (s,y) -> f s | y) (scopes,0)

let hhb = (po | union-scopes sso)+
\end{verbatim}
The function \texttt{union-scopes} takes a function~$f$ from scope
tags to relations as argument (as \texttt{sso} is)
and returns the union of the \texttt{$f(\textit{lvl})$} for all scope tags.
It refers to the \texttt{scopes} tag set, which is implicitly defined
by the \verb+enum scopes = +\ldots{} definition, and to the
\verb+fold+ (over sets) function, defined in appendix~\ref{coh}.
Figure~\ref{isa2sso} pictures the scope synchronization orders
for scopes agent and work-group, all transitive edges being omitted.
\begin{figure}
\caption{\label{isa2sso}Scope synchronization orders for scopes agent (\textsf{sso-agent}) and work-group (\textsf{sso-wg}).}
\begin{center}\hspace*{-2cm}\fmt{isa2+sso}\end{center}
\end{figure}
The figure also pictures the \hhb{} and \coh{} relations.

The HSA document defines three validity conditions on~\hhb.
The \hhb{} relation must be acyclic (equivalently irreflexive, as \hhb{} is
transitive), consistent with \coh{}, and consistent with sequentially consistent
orders (see next section).
We express the first two conditions as follows:
\begin{verbatim}
irreflexive hhb as HhbCons
call consistent (hhb,coh) as HhbCohCons
\end{verbatim}
Notice that Figure~\ref{isa2sso} demonstrates a case of inconsistency
of \hhb{} and~\coh{},
as the union of their pictured representations without transitivity edges is
cyclic. And, indeed, our \ltst{isa2} test
is similar\footnote{We have replaced \texttt{while} loops by \texttt{if} constructs.} to the test
``Race-free transitive synchronization through multiple scopes''
of the HSA document. It should thus be forbidden.

\section{Sequentially consistent synchronization order}
Sec. 3.10 of the HSA document states: ``\emph{there is a
total (apparent) order of all synchronizing operations with,
release, acquire, or acquire-release semantics in a single scope instance}''.
Given a scope instance~$S$, we write $\SCS{S}$ for this total order,
and $\SCLVL{lvl}$ for the union of  $\SCS{S}$ orders for all scope instances
at level~\textit{lvl}. We also abbreviate
sequentially consistent synchronization order as ``SC order''.

We have already defined the set \texttt{Synchronizing} in the previous section.
As to scope instances at level \textit{lvl} we represent then as the
equivalence relation~\symr{\it{lvl}}, which is available
in \cat{} from a scope tag as \texttt{tag2scope('\textit{lvl})}.
We here need a more direct representation of scope instances, as sets.
As a matter of fact,
for a given scope level~\textit{lvl}, scope instances at level~\textit{lvl}
are the equivalence classes of the relation~\symr{\itshape{}lvl}.
The \cat{} primitive \texttt{classes} takes an equivalence relation as argument
and returns its equivalence classes as a set of sets of events.
Hence we can compute the sets~$S$ for a given level as follows:
\marginpar{Luc for Jade, all our models use \texttt{same-instance} in place
of \texttt{tag2scope}\ldots}
\begin{verbatim}
let sync-instances(lvl) =
  classes ((Synchronizing * Synchronizing) & tag2scope(lvl))
\end{verbatim}

Then, the HSA  document clearly states that the total order $\SC{$S$}$
extends \po{}: ``\emph{Given synchronisation operations $X$ and~$Y$, if $X \po Y$ and $X$ and~$Y$ specify the same scope instance~$S$ (directly or indirectly
through inclusivity), then $X \SC{$S$} Y$}''.
However it is unclear if this applies to any pair of synchronizing operations
that belong to~$S$, or only to those whose scope annotations take effect
at the level of~$S$.
Nevertheless, as $\SC{$S$}$ is total and later required to be consistent
with $\po$ it does not matter much. We thus choose the first, more simple,
interpretation. Hence given a scope instance~$S$, we compute the set
of $\po$ linearisation on~$S$ as follows:
\begin{verbatim}
let preSC = po
let makeSCinstance(S) = linearisations(S,preSC)
\end{verbatim}
The \texttt{linearisation($S$,$r$)} primitive that computes all topological sorts of the graph $(S \times r)$ is introduced in appendix~\ref{coh}.
Finally, we compute the set of all possible $\SCLVL{lvl}$ relations as:
\begin{verbatim}
let makeSCscope(lvl) = cross (map makeSCinstance (sync-instances(lvl)))
\end{verbatim}
The function~\texttt{map} is map over sets. In the code above it serves
to compute the set of the sets of all possible $\SCS{S}$ orders
for all scope instances~$S$ at level~\textit{lvl}.
The function~\texttt{cross} takes a set of sets $\{S_1,S_2,\ldots,S_n\}$ as argument and returns the set of all sets built by picking one element
in each~$S_i$. Those two functions are introduced in appendix~\ref{coh}.

Finally, we iterate over all possible choices of $\SCLVL{lvl}$ for the five
HSA scope levels as follows:
\begin{verbatim}
with SWI from makeSCscope('wi)
call consistent(SWI,coh) as ScCohCons
call consistent(SWI,hhb) as ScHhbCons
with SWAVE from makeSCscope('wave)
call consistent(SWAVE,coh) as ScCohCons
call consistent(SWAVE,hhb) as ScHhbCons
with SWG from makeSCscope('wg)
call consistent(SWG,coh) as ScCohCons
call consistent(SWG,hhb) as ScHhbCons
with SAGENT from makeSCscope('agent)
call consistent(SAGENT,coh) as ScCohCons
call consistent(SAGENT,hhb) as ScHhbCons
with SSYSTEM from makeSCscope('system)
call consistent(SSYSTEM,coh) as ScCohCons
call consistent(SSYSTEM,hhb) as ScHhbCons
\end{verbatim}
Notice that we also check the consistency of SC orders with \coh{} and \hhb{},
as required, but not with~\po{}.
Indeed, by construction the ``order'' \SCLVL{lvl} includes~\po and
the two relation are thus consistent.

It remains to check that the SC orders are pairwise consistent:
\marginpar{Luc to Jade, unclear whether omitting the \texttt{ScSc}
checks will impact valid candidates or not.}
\begin{verbatim}
call consistent(SWI,SWAVE) as ScSc
call consistent(SWI,SWG) as ScSc
call consistent(SWI,SAGENT) as ScSc
call consistent(SWI,SSYSTEM) as ScSc
call consistent(SWAVE,SWG) as ScSc
call consistent(SWAVE,SAGENT) as ScSc
call consistent(SWAVE,SSYSTEM) as ScSc
call consistent(SWG,SAGENT) as ScSc
call consistent(SWG,SSYSTEM) as ScSc
call consistent(SAGENT,SSYSTEM) as ScSc
\end{verbatim}
As an example

\section{Races}
The HSA document provides two definition of conflicts, ordinary
and special.
Ordinary conflicts are defined as follows:
``\emph{Two operations $X$ and~$Y$ conflict, iff they access one or
more common byte locations, at least one is a write, and at least one is
an ordinary data operation}''.
Having noticed that ``conflicts'' form a symmetric relation, we paraphrase
the definition:
\begin{verbatim}
let at-least-one a = (a * _) | (_ * a)

let ordinary-conflicts = loc & at-least-one(W) & at-least-one(Ordinary)
\end{verbatim}
We use one pre-defined relation:
\verb+loc+ that relates accesses to the same location
(\herd{} does not handle mixed-size accesses yet),
and two pre-defined sets of events:
\verb+W+ the set of write operations, and \verb+Ordinary+ the set of
ordinary data operations.

The HSA document defines special conflicts as follows:
\begin{quote}\em
 Two special operations $X$ and~$Y$ conflict iff $X$ and $Y$ access
the same byte location and:
\begin{itemize}
\item[35.] $X$ and~$Y$ are different sizes (e.g., 32-bit vs. 64-bit), or
\item[36.] At least one is a write (or a read-modify-write),
and $\neg(\textit{Match}(\textit{SI}(X), \textit{SI}(Y))$.
\end{itemize}
\end{quote}
We ignore condition~35, by lack of mixed-size accesses,
and remind that \verb+W+, our pre-defined set of write operations,
includes the write performed by RMW operations.
Condition~36. refers to a (negated) \textit{Match} predicate
and to a~\textit{SI} function.
Both are defined in Sec. ``Scope instances'' of the HSA document.

The function~\textit{SI} returns the set of scope instances specified by
an operation, and the \textit{Match} predicate
tests the non-emptiness of intersection.
In our Sec.~\ref{sso} we have defined the relation~\texttt{same-instance('\textit{lvl})} that relates events that specify a common scope instance at level~\textit{lvl}. Hence, we represent~\textit{Match} by the relation~\verb+matches+
that relates events specifying a common scope instance at some level.
In effect, we quantify over scope levels rather than on pairs of operations
and  define \verb+matches+ as the union of \texttt{same-instance('\textit{lvl})}
for all scope levels. In \cat{} we write:
\begin{verbatim}
let matches = union-scopes same-instance

let special-conflicts =
  (loc & (Atomic * Atomic) & at-least-one(W)) \ matches
\end{verbatim}
The function~\verb+union-scopes+ that returns the union of
the application of a function on all scope tags is defined
in our Sec.~\ref{sso}.

The definition of HSA conflicts lacks a common additional condition:
accesses have to be by different units, which we consider in
the definition of conflicts below, by the means of the pre-defined~\verb+ext+
relation that relates operations by different units:
\begin{verbatim}
let conflicts = ((ordinary-conflicts|special-conflicts) & ext) \ at-least-one(I)
\end{verbatim}
We have also considered that initial writes (pre-defined set~\verb+I+)
do not enter in conflicts.
As a matter of fact \herd{} consider the initial value of a location
to come from an explicit initial write operation, for the $\rf$ relation to be
defined on all read operations.

We then define races as conflicts that are not ordered by HSA happens-before,
in either way:
\begin{verbatim}
let hsa-race = conflicts \ (hhb | hhb^-1)
\end{verbatim}
We used the postfix~$r\texttt{\^{}-1}$ operator that evaluates to
the inverse of relation~$r$.

It remains to inform the \herd{} simulator about race occurrence.
We do so with the \texttt{flag} construct, which apply to all checks.
The normal behaviour for a check is to stop model execution when invalid.
By contrast, a failing flagged check
does not stop execution but instead ``flags'' it with an arbitrary flag
(here \verb+undefined+).
Those flags are recorded and handed over to the \herd7{} machinery
at the end of model execution --- hence for valid executions that passed
all (unflagged) checks.
The simulator \herd{} can then decide that the simulated program is undefined,
as soon as one of the valid executions has been flagged as \verb+undefined+.
\begin{verbatim}
flag ~empty hsa-race as undefined
\end{verbatim}
Observe that the execution is flagged when the \verb+hsa-race+ relation is
\emph{not} empty.


\appendix
\section{\label{procedure}Checks and procedures}
A model text is a list of instructions.
Checks are significant instructions that, depending on check outcome,
will let execution continue or stop it.
For instance we can test that the two relations
\coh{} and \po{} are consistent by the instruction:
\begin{verbatim}
irreflexive coh;po as CohPoCons
\end{verbatim}
Here, execution will continue when \coh{} and~\po{} are consistent,
\emph{i.e.} when the sequence relation $\coh; \po$ is irreflexive.
Further notice that checks have an optional name, introduced by the \texttt{as}
keyword, for documentation and control purposes (see \herd{} option \texttt{-skipchecks}).

In the case of the consistency check it may be interesting to abstract the
details of the check by defining a procedure:
\begin{verbatim}
procedure consistent(a,b) =
  irreflexive a;b
end
\end{verbatim}

A procedure is called by using the explicit ``\texttt{call}'' keyword.
As an example the consistency of \coh{} and~\po{} can be checked as follows:
\begin{verbatim}
call consistent(coh,po) as CohPoCons
\end{verbatim}


\section{\label{coh}Generating \coh}

The ``coherence'' relation is a fairly classical one.
It can be generated by \herd{} from the $\coinit$ pre-defined relation.
This relation expresses a few constraints on the writes:
it relates initial writes to all other writes and
all non-final writes to final writes, for each  location~$L$.
Those constraints on writes are introduced by \herd{} initial machinery
that enumerates all candidate executions, considering all possible
final writes\footnote{As a matter of fact, we assume that each location
holds a well defined value at the end of program execution, and we
enumerate all such final values.} in turn.

In \herd{}, we can generate all orders on a certain set of events with the
\texttt{linearisations($S$,$R$)} primitive that takes two arguments:
a set of events~$S$ and a relation~$R$.
The primitive considers $R_S$, the restriction of~$R$ to $S$ (written
\texttt{$R$ \& ($S$ * $S$)} in the \cat{} language)
If $R_S$ is acyclic, the call will return the set of all total orders
that extend $R_S$. Otherwise, \emph{i.e.} if $R_S$ has a cycle,
the primitive returns the empty set.
Hence, assuming $S_L$ to be the set of all memory events to location~$L$,
one can generate the set of all possible $\cohl{L}$ by the call
\texttt{linearisations($S_L$,co0)}.
\begin{verbatim}
let makeCohL(s) = linearisations(s,co0)
\end{verbatim}

In fact, we want to generate the set of all possible $\coh$ relations,
\emph{i.e.} all the unions of all the possible $\cohl{L}$ orders for all
locations~$L$. To that end we use another \cat{} primitive:
\texttt{partition($S$)} that takes a set of events as argument and
returns a set of set of events $T = \{S_1,\ldots,S_n\}$, where each
$S_i$ is the set of all events in $S$ that act on location $L_i$,
and, of course $S$ is the union $\bigcup_{i=1}^{i=n} S_i$.

Combining the effect of the \texttt{partition} and \texttt{linearisations}
primitives is possible in the \cat{} language.
We first define a, quite classical,
\texttt{map} functions that, given a set~$S= \{e_1,\ldots,e_n\}$
and a function $f$, returns the set $\{f(e_1),\ldots,f(e_n)\}$:
\begin{verbatim}
let map f  =
  let rec do_map es = match es with
  || {} -> {}
  || e ++ es -> f e ++ do_map es
  end in
  do_map
\end{verbatim}
Notice that \texttt{map} is written in curried style.
The code above uses a few of \cat{} set and relation construct
``\texttt{++}'' is set addition (in infix notation),
``\verb+{}+ is the empty set.
It also uses set pattern matching that permits recursion over sets,
by considering the empty and non-empty cases, in a manner similar
to list pattern matching.

Then, we generate the set of all possible \cohl{L} for all locations~$L$ as
follows:
\begin{verbatim}
let allCohL = map makeCohL (partitions(M))
\end{verbatim}

Now, \texttt{allCohL} is a set of set of relations, one element of which
is the set of all possible $\cohl{L}$ orders for a specific~$L$.
It remains to generate all possible unions of the $\cohl{L}$ for
all possible choices of those and all possible locations~$L$.
It can be done by another \cat{} function~\texttt{cross}, that takes
a set of sets $S = \{S_1, S_2, \ldots, S_n\}$
as argument and returns all possible unions
built by picking elements from each of the $S_i$:
$$
\left\{\, e_1 \cup e_2 \cup \cdots \cup e_n \mid
e_1 \in S_1, e_2 \in S_2, \ldots, e_n \in S_n \,\right\}
$$
\begin{verbatim}
let fold f =
  let rec fold_rec (es,y) = match es with
  || {} -> y
  || e ++ es -> fold_rec (es,f (e,y))
  end in
  fold_rec

let rec cross ess = match ess with
  || {} -> { 0 }
  || es ++ ess ->
      let yss = cross ess in
      fold
        (fun (e,r) -> map (fun ys -> e | ys) yss | r)
        (es,{})           
  end      
\end{verbatim}
In the code above,
one may notice the use of the union operator ``\verb+|+''
and of the empty relation ``\verb+0+''.

Finally we generate all possible $\coh$ ``orders'' by:
\begin{verbatim}
let allCoh = cross allCohL
\end{verbatim}

\section{\label{bell}The bell file for HSA}

\end{document}
