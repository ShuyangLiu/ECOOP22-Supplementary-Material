\section{Instructions and events}

Instructions can be reads, writes, read-modify-writes, fences and branches.
Instructions can bear \emph{tags}, for example relative to their \emph{kind},
\emph{memory order}, (i.e. the ordering constraints they induce), or
\emph{scope}.

Events give a semantics to instructions; they can be accesses (reads or writes,
fences or branches. We model read-modify-write instructions with two events: a
read and a write. Events bear the same tags as the instructions they come
from.

\subsection{Tags}

\subsubsection{Kind}

An event can be \emph{ordinary}, or \emph{atomic}, and bear the eponymous tag:
\begin{verbatim}
enum kind = 'ordinary || 'atomic
\end{verbatim}

%\subsubsection{Size}
%
%An event can be single-copy atomic, in which case it is tagged with {\tt
%single} or not, in which case it is tagged with {\tt double}:
%\begin{verbatim}
%enum size = 'single || 'double
%\end{verbatim}
%
%Having the same size simply corresponds to bearing the same size tag:
%\begin{verbatim}
%let same-size = (Single*Single) | (Double*Double)
%\end{verbatim}

\subsubsection{Memory order}

Memory orders are tags that induce certain ordering properties to the
events which bear them. Events can be \emph{relax} (tag {\tt rlx}),
\emph{acquire} (tag {\tt scacq}), \emph{release} (tag {\tt screl}), or both
release and acquire (tag {\tt scar}):
\begin{verbatim}
enum memory-order = 'rlx || 'scacq || 'screl || 'scar
\end{verbatim}

%{\color{blue} We have several questions on the subject of memory order tags:
%\begin{itemize}
%\item there seems to be little difference (if any) between a read operation
%annotated with scacq and a read operation annotated with scar (the same remark
%applies to a write operation annotated with screl vs. scar). What is the
%purpose of scacq and screl? For now we find they seem to act the same as scar
%for reads and writes, respectively.
%\item We model RMWs with a read and a write. Therefore we wonder what the
%meaning of the tag scacq can be a RMW, as it is illegal for a write. The same
%remark applies to tagging an RMW with screl, as screl is illegal for a read.
%\end{itemize}
%}

\subsubsection{Scopes}

Scope tags reflect the concurrency hierarchy; work-item events are tagged {\tt
wi}, and work-group events {\tt wg}, the other tags being eponymous of the
scope level: 
\begin{verbatim}
enum scopes = 'wi || 'wave || 'wg || 'agent || 'system
\end{verbatim}

We go back to scopes in Section~\ref{sec:scopes}

\subsection{Accesses}

Accesses can be reads, writes or read-modify-writes, and bear various
combinatinons of the tags defined above.

\subsubsection{Reads}

Reads can be either ordinary or atomic, and this influences what memory order
and scope tags they bear:
\begin{verbatim}
default R[{'ordinary},{'rlx},{'wi}]
instructions R[{'atomic},{'rlx,'scacq,'scar},scopes]
\end{verbatim}

Thus reads can be ordinary, in which case they have to be relaxed and at scope
work-item. %; ordinary reads can be single-copy atomic or not. {\color{blue} Or
%do they have to be single-copy atomic?}

Atomic reads can be relaxed, acquire, or release-acquire, and live at any scope
level. %, with any size.

\subsubsection{Writes}

Writes can be either ordinary or atomic, and this influences what memory order
and scope tags they bear:
\begin{verbatim}
default W[{'ordinary},{'rlx},{'wi}]
instructions W[{'atomic},{'rlx,'screl,'scar},scopes]
\end{verbatim}

Thus writes can be ordinary, in which case they have to be relaxed and at scope
work-item. %; ordinary writes can be single-copy atomic or not. {\color{blue} Or
%do they have to be single-copy atomic?}

Atomic writes can be relaxed, acquire, or release-acquire, and live at any scope
level. %, with any size.

\subsubsection{Read-modify-writes}

Read-modify-writes have to be atomic, and can bear any memory order, scope
tags:
%or size tags:
\begin{verbatim}
instructions RMW[{'atomic},memory-order,scopes]
\end{verbatim}

\subsubsection{Fences}

Fence events can bear memory order or scope tags:
\begin{verbatim}
instructions F[{'scacq,'screl,'scar},scopes]
\end{verbatim}

%\section{\label{segment}Segments}
%
%{\color{blue} We have several questions on the subject of memory segments:
%\begin{itemize}
%\item In the documentation, example ``Race­free transitive synchronization
%through multiple scopes'': we are a bit surprised by the fact that variable Z
%is defined in the group segment, as it is used for synchronisation between
%units B and C, which belong to different groups. Is that intentional?
%\item In the documentation, example ``Separate Segment synchronisation'': the
%program declares X as global and Y as group. Essentially, the test seems to be
%a release-acquire idiom using Y as flag and X as data. Hence we're not too sure
%how to understand the final comment: ``This example shows that synchronization
%can cross segments. Even though the atomic store  and load specify a location
%in global memory, they still synchronize the group location Y.'' Are the atomic
%store and load the ones over X, or Y?
%\item Moreover, about the same example, the group variable Y is initialised to
%$0$, whereas the HSAIL documentation seems to say that one cannot initialise
%group variable; is that ok?
%\end{itemize} }

%We summarise the possible annotations of events in \mytab\ref{fig:annotations}.
%
%\begin{table}[!h]
%  \centering
%\scalebox{.81}{
%\begin{tabular}{c|c|c}
%         & accesses                   & fences              \\\hline
%kind     & {\tt }      & {\tt }         \\
%scope    & {\tt }        & {\tt }  
%\end{tabular}}
%\caption{Possible annotations of events \label{fig:annotations}}
%\end{table}

\section{Concurrency hierarchy \label{sec:scopes}}

%\begin{figure}[!h]
%\vspace*{-4mm}
%\begin{center}
%%\scalebox{.6}{\includegraphics{hierarchy}}
%\end{center}
%\vspace*{-2mm}
%\caption{Concurrency hierarchy\label{fig:hierarchy}}
%\vspace*{-6mm}
%\end{figure}

An HSA system is hierarchical; here are the scopes that we consider, each
\narrower{} than the following one (resp. \wider{} than the previous one):
\begin{itemize}
\item {\tt work item};
\item {\tt wave};
\item {\tt work group};
\item {\tt agent};
\item {\tt system}. 
\end{itemize} 

Our \prog{herd} tool handles scopes by the means of various objects: scope
tags, scope trees, scope instances and sets of tagged events. These objects are
all defined by reading a prelude {\tt bell} file, supplied to \herd{} with the
\texttt{-bell} option. Section~\ref{sec:bell} details the {\tt bell} file for
HSA.

\subsection{Scope tags}
HSA defines a hierarchy of five scope levels: system, agent, work-group, wave
and work-item here listed from wider to narrower.

Scope tags reflect these levels; work-item events are tagged {\tt wi}, and
work-group events {\tt wg}, the other tags being eponymous of the scope level:
\begin{verbatim}
enum scopes = 'wi || 'wave || 'wg || 'agent || 'system
\end{verbatim}

\subsubsection{Concurrency hierarchy}

The functions {\tt narrower} and {\tt wider} implement the concurrency
hierarchy, making for example the scope level {\tt agent} narrower than {\tt
system}, and the scope level {\tt wave} wider than work-item {\tt wi}:
\begin{verbatim}
let narrower(s) = match s with
  || 'system -> 'agent
  || 'agent -> 'wg
  || 'wg -> 'wave
  || 'wave -> 'wi
end

let wider(s) = match s with
  || 'agent -> 'system
  || 'wg -> 'agent
  || 'wave -> 'wg
  || 'wi -> 'wave
end
\end{verbatim}

In these two functions we use the {\tt cat} construct {\tt match}; one can use
{\tt match} to match a tag value (here the argument {\tt s}) against a set of
\emph{patterns} (here the enum type of scope levels), and for each pattern
return a result.

\subsubsection{Sets of tagged events}

Scope tags can annotate events. For each tag {\tt 't}, our \prog{herd} tool
builds the set {\tt T} of all events that bear the tag {\tt 't}. For instance,
in Figure~\ref{isa2scopes}, we have
$\texttt{Wg} = \{ b \}$,
$\texttt{Agent} = \{ c \}$,
$\texttt{System} = \{ d,e \}$.
Sets of tagged events can be accessed from the scope tag
by a primitive, \verb+tag2set+.  For instance, in our \ltst{isa2} example, the
expression \verb+tag2set('system)+ will evaluate to the set of events
$\{d,e\}$.

\subsection{Scope trees}

Each test features a ``scope tree'' definition that specifies how the test
units are organised according to the scope hierarchy (defined by the fuctions
{\tt wider} and {\tt narrower}). Consider for instance the test \ltst{isa2} (or experiment with it at \url{virginia.cs.ucl.ac.uk/?record=hsa&cat=hsa&bell=hsa&litmus=isa2}):

\pagebreak

{\footnotesize
\verbatiminput{img/isa2+scopes.litmus}
}
This test has three units, \myth{0}, \myth{1} and~\myth{2}.
According to the scope tree \verb+(agent (wg 0 1) (wg 2))+,
the first two units \myth{0} and \myth{1} are in the same work-group,
while the last unit~\myth{2} is in a different work-group.
Then, all units are in the same agent.
Our notation for scope trees is abbreviated by omitting
the scope tree levels that contain exactly one sub-instance.
Here, the complete scope tree specification is:
\begin{verbatim}
(system (agent (wg (wave (wi 0)) (wave (wi 1))) (wg (wave (wi 2)))))
\end{verbatim}
Observe that, in the complete scope tree above,
the system scope contains exactly one agent, each work-group contain
exactly one wave, etc.

\subsection{Scope instances}
Hence, at runtime, there will be two ``scope instances'' of the work-group
level. We model these scope instances in \cat{} as relations eponymous of the
corresponding scope levels. Thus a scope instance is an equivalence relation
that relates events whose units are in the same scope instance of the given
level.

\subsubsection{Scope instances}
Given a tag \texttt{'t}, \texttt{tag2scope('t)} returns the eponymous scope
instance relation {\tt t}.

As an example, Figure~\ref{isa2scopes} pictures the \wg{} scope instance (or
{\tt tag2scope('wg)}) for the \ltst{isa2} test.
We have omitted the reflexivity edges (\emph{i.e.} $a \wg a$, $b \wg b$ etc.) for clarity.
\begin{figure}[!h]
\begin{center}\moveback
\fmt{isa2+scopes}
\end{center}
\caption{\label{isa2scopes} The \wg{} scope instance of the \ltst{isa2} test \color{green}{(Allowed)}}
\end{figure}

We could also have pictured the \agent{} scope instance. We refrain from doing
so, as there is a single agent scope instance that contains all units.  Thus,
the \agent{} scope instance in this case is the total relation over events, and
picturing it would clobber the diagram.

%{\color{blue} Is that right? More precisely, does our notion of scope instance
%match the intent?}

\subsubsection{Same instance}

Section 3.9 ```Scoped synchronization order'' of the HSA document states that
two operations may ''\emph{both specify (directly or indirectly through scope
inclusion) scope instance~$S$}''. We believe that this means for example that a
memory operation with tag agent also specifies work-group. More generally, the
scope tag~\text{tag} takes effect on all scopes at levels {\tt narrower}
than~\text{tag}. %{\color{blue} Is that right?}

To handle this bit of the model, we define equivalence relations
\texttt{same-\text{tag}}, where \text{tag} is a scope tag. Two operations
are related by \texttt{same-\text{tag}} when
\begin{itemize}
\item the two operations are related by the scope instance \text{tag}, but
also
\item they are tagged by scope~\text{tag} or {\tt wider}.
\end{itemize}

Hence, we first define a function \texttt{all-instrs} that takes
a (scope) tag as argument and returns all the events annotated by
this scope or {\tt wider}:

\begin{verbatim}
let rec all-instrs(tag) = match tag with
|| 'system -> tag2instrs(tag)
|| _ -> tag2instrs(tag) | all-instrs(wider(tag))
end
\end{verbatim}

Here we use {\tt let rec} because the function {\tt all-instrs} is
\emph{recursive} (see how {\tt all-instrs} is called within itself).

Now we can define the function~\texttt{do-same-instance} that takes a scope
tag~\texttt{'\text{tag}} as argument and returns the
relation~\texttt{same-\text{tag}}:
\begin{verbatim}
let do-same-instance(tag) =
 let instrs = all-instrs(tag) in
 tag2scope(tag) & (instrs * instrs)
\end{verbatim}
The function {\tt same-instance} simply computes the intersection of the scope
instance {\tt tag2scope(tag)} and of the cartesian product {\tt (instrs *
instrs)} on events tagged with scope level~\text{tag} or {\tt wider}.

\begin{figure}[!h]
\begin{center}\moveback
\fmt{isa2+instances}
\end{center}
\caption{\label{isa2same}The \same{wg} and~\same{agent} relations of the \ltst{isa2} test \color{green}{(Allowed)}}
\end{figure}
Figure~\ref{isa2same} shows two {\tt same-\text{tag}} relations (once again
we omitted reflexivity edges for clarity; in particular $e \same{wg} e$ is
omitted).  Also note that we both have $c \same{agent} d$ and~$c \same{wg} d$.
This comes from scope inclusion: events $c$ and~$d$ are from the same
unit~\myth{1}.  Hence, they belong to the same agent and work-group scope
instances.  Moreover, given event scope annotations (\texttt{agent} and
\texttt{system} respectively) the events both act up to the agent level.

%\subsubsection{A final question}
%
%{\color{blue} Consider the following excerpt, in the assembly given in the HSA
%documentation, where A and B are in different work groups:
%\begin{verbatim}
%group_u32 &X ;
%
%A:
%atomic_st_scar_wg_u32 [&X],1 ;
%
%B:
%atomic_st_scar_wg_u32 [&X],2 ;
%\end{verbatim}

%We wonder if we should consider that there is one variable $X$, or rather two,
%one for each scope instance. Another way of seeing this question is to
%determine whether the program is racy or not: with one variable $X$ it is racy,
%otherwise it isn't.}

%\subsection{Conventions} 
%
%We use several notations that rely on \emph{relations} and \emph{orders}. We
%write {\tt 0} for the empty relation.  We denote the transitive
%(resp.~reflexive-transitive) closure of a relation $\textsf{r}$ as
%$\transc{\textsf{r}}$ (resp.~$\rstar{\textsf{r}}$).  We write
%$\textsf{r}_1;\textsf{r}_2$ for the sequential composition of two relations
%$\textsf{r}_1$ and $\textsf{r}_2$, \ie $\dfn{(x,y) \in
%(\textsf{r}_1;\textsf{r}_2)}{\exists z.  (x,z) \in \textsf{r}_1 \:\:\wedge}$
%$(z,y) \in \textsf{r}_2$. We write $\irrefl(\textsf{r})$ to express the
%irreflexivity of $\textsf{r}$, \ie $\neg(\exists x. (x,x) \in \textsf{r})$. We
%write $\acyclic(\textsf{r})$ to express its acyclicity, \ie the irreflexivity
%of its transitive closure: $\neg(\exists x.  (x,x) \in \transc{\textsf{r}})$.
%
%A \emph{partial order} is a relation $\textsf{r}$ that is \emph{transitive}
%(\ie $\textsf{r} = \transc{\textsf{r}}$), and irreflexive. Note that this
%entails that $\textsf{r}$ is also acyclic. A \emph{total order} is a partial
%order $\textsf{r}$ defined over a set $\mathbb{S}$ that enjoys the
%\emph{totality property}: $\forall x \neq y \in \mathbb{S}. (x,y) \in
%\textsf{r} \vee (y,x) \in \textsf{r}$.
%
%We refer to orderings of events \wrt several relations.  To avoid ambiguity,
%given a relation \textsf{r}, we say that an event $e_1$ is
%\emph{\textsf{r}-before} another event $e_2$ (or $e_1$ is an
%\emph{\textsf{r}-predecessor} of $e_1$, or $e_2$ is \emph{\textsf{r}-after}
%$e_1$, or $e_2$ is \emph{\textsf{r}-subsequent}, \etc) when $(e_1,e_2) \in
%\textsf{r}$.
%
%\subsection{Executions} Executions are tuples $(\evts,\po,\rf,\co)$, which
%consist of a set of \emph{events} $\evts$, giving a semantics to the
%instructions, and three relations over events: $\po,\rf$ and $\co$ (see below).
%Note that our executions are post-mortem.


%\paragraph{Auxiliaries over events}
%Given a set of events, we write {\tt W*R, W*W, R*R, R*W} for the set of
%write-read, write-write, read-read and read-write pairs respectively. We write
%{\tt po \& W*R} for the write-read pairs in program order, and {\tt po
%$\setminus$ W*R} for all the pairs in program order except the write-read
%pairs. We write {\tt R*M} for the set of read-read and read-write pairs ({\tt
%M} designates the set of all memory actions, i.e. the union of {\tt R} and {\tt
%W}).

\section{Relations over events} 

\subsection{Program order} 
The \emph{program order} $\po$ lifts the order in which instructions have been
written in the program to the level of events. The program order is a total
order over the memory events of a given thread, but does not order events from
different threads. Note that program order unrolls loops and determines the
branches taken. 

\subsection{Read-from}
The \emph{read-from} $\rf$ links a read from a register or a memory location to
a unique write to the same register or location. The value of the read must be
equal to the one of the write. We write $\rfe$ (external read-from) when the
events related by $\rf$ belong to distinct threads, \ie $\dfn{(w,r) \in
\rfe}{(w,r) \in \rf \wedge \pr(w) \neq \pr(r)}$. We write $\rfi$ for internal
read-from, when the events belong to the same thread.

%\subsection{Coherence \label{sec:co}}
%The \emph{coherence} order $\co$ totally orders writes to the same memory
%location. We write $\coi$ (resp.~$\coe$) for internal (resp.~external)
%coherence.

%\subsection{From-read}
%We derive the \emph{from-read} $\fr$ from the read-from $\rf$ and the coherence
%$\co$, as follows:
%\begin{center}
%%\newfmt{fr}\vspace*{-2ex}
%\end{center}
%That is, a read $r$ is in $\fr$ with a write $w_1$ (resp.~$w_2$) if $r$ reads
%from a write $w_0$ such that $w_0$ is in the coherence order before $w_1$
%(resp.~$w_2$).  We write $\fri$ (resp.~$\fre$) for the internal
%(resp.~external) from-read.

%\subsection{Communications}
%We gather all \emph{communications} in $\dfn{\com}{\co \cup \rf \cup \fr}$. 

%\subsection{Glossary}
%We give a glossary of all the relations that we describe in this section
%in~\mytab\ref{fig:gloss-rlns}. For each relation we give its notation, its name
%in English, the directions (\ie write W or read R) of the source and target of
%the relation (column ``dirns''), where to find it in the text (column
%``reference''), and an informal prose description.  Additionally in the column
%``nature'', we give a taxonomy of our relations: are they fundamental execution
%relations (\eg $\po, \rf$), architectural relations (\eg $\ppo$), or derived
%(\eg $\fr$)?

%\begin{table}[!h]
%  \centering
%\scalebox{.81}{
%\begin{tabular}{c|p{.2\linewidth}|c|l|p{.17\linewidth}|p{.3\linewidth}}
%notation & name          & nature & dirns & reference & description \\\midrule
%$\po$    & program order & execution & any, any & \S Relations over events & instruction order lifted to events \\[.3em]
%$\rf$    & read-from     & execution & WR & \S Relations over events & links a write $w$ to a read $r$ taking its value from $w$ \\[.3em]
%$\co$    & coherence     & execution & WW & \S Relations over events & total order over writes to the same memory location \\\midrule
%%$\ppo$   & preserved program order & architecture & any, any & \S Architectures & program order maintained by the architecture \\[.3em]
%%$\ffence, \ff$ & full fence & architecture & any, any & \S Architectures        & \eg  \\[.3em]
%%$\lwfence, \lwf$ & lightweight fence & architecture & any, any & \S Architectures & \eg  \\[.3em]
%%$\fences$ & fences & architecture & any, any & \S Architectures & architecture-dependent subset of the fence relations, \eg $\ffence, \lwfence$ \\[.3em]
%%$\prop$ & propagation & architecture & & \S Architectures
%%  & order in which writes propagate, typically enforced by fences
%%  \\\midrule
%$\poloc$ & program order restricted to the same memory location & derived & any, any & \S \textsc{sc per location} & $\{(x,y) \mid (x,y) \in \po \wedge \loc(x)=\loc(y)\}$\\[.3em]
%$\com$ & communications & derived & any, any &  \S Relations over events            & $\co \cup \rf \cup \fr$ \\[.3em]
%$\fr$    & from-read   & derived  & RW & \S Relations over events & links a read $r$ to a write $w'$ $\co$-after the write $w$ from which $r$ takes its value \\[.3em]
%%$\cause(ssc)$ & causality & derived & any, any & \S \textsc{no thin air} & $\ppo \cup \fences \cup \rfe \cup \cause(\narrowerop(ssc))$ \\[.3em]
%\end{tabular}}
%\caption{Glossary of relations\label{fig:gloss-rlns}}
%\end{table}

%\pagebreak
%
%\subsection{A first litmus test}
%
%In the following we present several examples of executions. We depict the
%events of a given thread vertically to represent the program order, and the
%communications by arrows labelled with the corresponding relation.
%
%\subsubsection{Message passing}
%\myfig\ref{fig:mp-naked} shows a classic \emph{message passing} (\textsf{mp})
%example. This example is a communication pattern involving two memory locations
%$x$ and $y$: $x$ is a message, and $y$ a flag to signal to the other thread
%that it can access the message. 
%\begin{figure}[!h]
%\begin{center}
%%\newfmt{mp}
%\end{center}
%\vspace*{-10mm}
%\caption{Message passing pattern\label{fig:mp-naked}}
%\end{figure}
%
%\myth{0} writes the value~$1$ to memory at location~$x$ (see the event~$a$).
%In program order after~$a$ (hence the~$\po$ arrow between~$a$ and~$b$), we have
%a write of value~$1$ to memory at location~$y$. \myth{1} reads from~$y$ (see
%the event~$c$). In the particular execution shown here, this read takes its
%value from the write~$b$ by~\myth{0}, hence the~$\rf$ arrow between~$b$
%and~$c$. In program order after~$c$, we have a read from location~$x$. In this
%execution, we suppose that this event~$d$ reads from the initial state (not
%depicted), which by convention sets the values in all memory locations and
%registers to~$0$.  This is the reason why the read~$d$ has the value~$0$. This
%initial write to~$x$ is, by convention, $\co$-before the write~$a$ of~$x$
%by~\myth{0}, hence we have an~$\fr$ arrow between~$d$ and~$a$.
%
%Note that, in the following, even if we do not always depict all of the program
%order, a program order edge is always implied between each pair of events
%ordered vertically below a thread id, \eg $\textsf{T}_0$.
%
%\subsubsection{Convention for naming tests} We refer to tests as follows; we
%roughly have two flavours of names: classical names, which are abbreviations of
%classical litmus test names appearing in the literature; and systematic names,
%which describe the accesses occurring on each thread of a test.
%
%Classical patterns, such as the message passing pattern above, have an
%abbreviated name: \textsf{mp} stands for ``message passing'', \textsf{sb} for
%``store buffering'', \textsf{lb} for ``load buffering'', \textsf{wrc} for
%``write-to-read causality'', \textsf{rwc} for ``read-to-write causality''.
%
%\begin{table}[!h]
%\begin{tabular}{c|c|c|p{.55\linewidth}}
%classic & systematic & diagram & description \\\midrule
%\textsf{coXY} & & \myfig\ref{fig:co} and \ref{fig:corr} & coherence test
%involving an access of kind X and an access of kind Y; X and Y can be either R
%(read) or W (write)\\[.3em]
%\textsf{lb} & \textsf{rw+rw} & \myfig\ref{fig:lb} & load buffering \ie two threads each holding a read then a write \\[.3em]
%\textsf{mp} & \textsf{ww+rr}& \myfig\ref{fig:mp} & message passing \ie two threads; first thread holds two writes, second thread holds two reads \\[.3em]
%\textsf{wrc} & \textsf{w+rw+rr} & \myfig\ref{fig:wrc} & write to read causality \ie three threads; first thread holds a write, second thread holds a read then a write, third thread holds two reads \\[.3em]
%\textsf{isa2} & \textsf{ww+rw+rr} & \myfig\ref{fig:isa2} & one of the tests appearing in the Power ISA documentation \ie write to read causality prefixed by a write, meaning that the first thread holds two writes instead of just one as in the \textsf{wrc} case \\[.3em]
%\textsf{2+2w} & \textsf{ww+ww} & \myfig\ref{fig:2+2w} & two threads holding two writes each \\[.3em]
%& \textsf{w+rw+2w} & \myfig\ref{fig:w+rw+2w} & three threads; first thread
%holds a write, second thread holds a read then a write, third thread holds two
%writes \\[.3em]
%\textsf{sb} & \textsf{wr+wr} & \myfig\ref{fig:sb} & store buffering \ie two threads each holding a write then a read \\[.3em]
%\textsf{iriw} & \textsf{w+rr+w+rr} & \myfig\ref{fig:iriw} & independent reads of independent writes \ie four threads; first thread holds a write, second holds two reads, third holds a write, fourth holds two reads  \\[.3em] 
%\textsf{rwc} & \textsf{w+rr+wr} & \myfig\ref{fig:rwc} & read to write causality over three threads; first thread holds a write, second thread holds two reads, third thread holds a write then a read \\[.3em]
%\textsf{r} & \textsf{ww+wr} & \myfig\ref{fig:r} & two threads; first thread
%holds two writes, second thread holds a read and a write
%\end{tabular}
%\caption{Glossary of litmus tests names\label{fig:gloss-litmus}}
%\end{table}

%When a pattern does not have a classical name from the literature, we give it a
%name that simply describes which accesses occur: for example \textsf{2+2w}
%means that the test is made of two threads holding two writes each;
%\textsf{w+rw+2w} means that we have three threads: a write on a first thread, a
%read followed by a write on a second thread, and then two writes on another
%thread.

%We give a glossary of the test names presented in this paper in
%\mytab\ref{fig:gloss-litmus}, in the order in which they appear; for each test
%we give its systematic name, and its classic name (\ie borrowed from previous
%works) when there is one.

%Given a pattern such as \textsf{mp} above, we write \textsf{mp+fence+ppo} for
%the same underlying pattern where in addition the first thread has a
%lightweight fence \lwfence{} between the two writes and the second thread
%maintains its two accesses in order thanks to some \emph{preserved program
%order} mechanism (\ppo, see below). We write \textsf{mp+fences} for the
%\textsf{mp} pattern with two lightweight fences, one on each thread. 
%

\pagebreak

%\subsection{Architectures} Architectures are instances of our model.  An
%architecture is a triple of functions $(\ppo,\fences,\prop)$, which specifies
%the \emph{preserved program order} $\ppo$, the \emph{fences} $\fences$ and the
%\emph{propagation order} $\prop$.

%\smallskip
%
%\subsubsection{Preserved program order\label{sec:preamble:ppo}}
%The preserved program order gathers the set of pairs of events which are
%guaranteed not to be reordered \wrt the order in which the corresponding
%instructions occur in the program text. Thus two events $e_1$ and $e_2$ being
%in program order (i.e. $(e_1,e_2) \in \po$) does not necessarily entail their
%being in preserved program order. 
%
%For example on TSO, only write-read pairs can be reordered, so that the
%preserved program order for TSO is $\po \setminus W*R$. On weaker models such
%as Volta, the preserved program order merely includes \emph{dependencies}, for
%example address dependencies, when the address of a memory access is determined
%by the value read by a preceding load.  We detail these notions, and the
%preserved program order for Volta, in \mysec\ref{sec:ppo}.
%
%The function $\ppo$, given an execution $(\evts,\po,\co,\rf)$, returns the
%preserved program order. For example, consider the execution of the message
%passing example given in \myfig\ref{fig:mp}.  Assume that there is an address
%dependency between the two reads on \myth{1}. As such a dependency constitutes
%a preserved program order relation on Volta, the $\ppo$ function would return
%the pair $(c,d)$ for this particular execution.
%
%\smallskip
%
%\subsubsection{Fences}
%Fences (or \emph{memory barriers}) are special instructions which prevent
%certain behaviours.  
% 
%In this document, we use the same names for the fence instructions and the
%relations that they induce over events. For example, consider the execution of
%the message passing example given in \myfig\ref{fig:mp}.  Assume that there is
%a lightweight fence \lwfence{} between the two writes $a$ and $b$ on \myth{0}.
%In this case, we would have $(a,b) \in \lwfence{}$. Note however, that just
%like program order does not entail preserved program order (see
%\mysec\ref{sec:preamble:ppo}), two events $e_1$ and $e_2$ being separated (in
%program order) by a fence does not necessarily entail that $e_1$ and $e_2$ will
%benefit from ordering properties due to the fence. Formally, the pair
%$(e_1,e_2)$ might not be included in the \emph{propagation order}, which we
%expose below in \mysec\ref{sec:preamble:prop}.
%
%The function $\fences$ returns the pairs of events in program order which are
%separated by a fence, when given an execution. For example, consider the
%execution of the message passing example given in \myfig\ref{fig:mp}.  Assume
%that there is a lightweight Volta fence \lwfence{}  between the two writes on
%\myth{0}. On Volta, the $\fences$ function would thus return the pair $(a,b)$
%for this particular execution.
%
%\smallskip
%
%\subsubsection{Propagation order\label{sec:preamble:prop}}
%The propagation order constrains the order in which writes are propagated to
%the memory system. This order is a partial order between writes (not
%necessarily to the same location), which can be enforced by using fences. For
%example on Volta, two writes in program order separated by an \lwfence{}
%barrier (see \myfig\ref{fig:mp}) will be ordered the same way in the
%propagation order.

%We note that the propagation order is distinct from the coherence order $\co$:
%indeed $\co$ only orders writes to the same location, whereas the propagation
%order can relate writes with different locations through the use of fences.
%However, both orders have to be compatible, as expressed by our
%\textsc{propagation} axiom, which we explain next (see
%\myfig\ref{fig:model-axioms} and \myfig\ref{fig:2+2w}). 

%The function $\prop$ returns the pairs of writes ordered by the propagation
%order, given an execution. For example, consider the execution of the message
%passing example given in \myfig\ref{fig:mp}.  Assume that there is a
%lightweight Volta fence \lwfence{}  between the two writes on \myth{0}. On
%Volta, the presence of this fence forces the two writes to propagate in the
%order in which they are written on \myth{0}. The function $\prop$ would thus
%return the pair $(a,b)$ for this particular execution.
